%%%%%%%%%%%%%%%%%%%%%%%%%%%%%%%%%%%%%%%%%%%%%%%%%%%%%%%%%%%%%%%%%%%%%%%%%
%%
%W  basics.tex      ANUPQ documentation - background info   Werner Nickel
%W                                                      Joachim Neubueser
%%
%H  $Id$
%%
%%

%%%%%%%%%%%%%%%%%%%%%%%%%%%%%%%%%%%%%%%%%%%%%%%%%%%%%%%%%%%%%%%%%%%%%%%%%
\Chapter{Mathematical Background and Terminology}

In this chapter  we will give a brief  description of the mathematical
notions used in the algorithms  implemented in the ANU `pq' program
that are made accessible from {\GAP} through this package.  For proofs
and  details  we  will  point  to relevant  places  in  the  published
literature.  Also we will try  to give some explanation of terminology
that may help to use the ``low-level'' interactive functions described
in Section~"Low-level Interactive ANUPQ  Functions based on menu items
of the pq program".

%%%%%%%%%%%%%%%%%%%%%%%%%%%%%%%%%%%%%%%%%%%%%%%%%%%%%%%%%%%%%%%%%%%%%%
\Section{Basic notions}

*pc Presentations and Consistency*

For details, see  e.g.~\cite{NNN98}.

Every $p$-group $G$ has a presentation of the form: 
$$
\{a_1,\dots,a_n \mid a_i^p = v_{ii}, 1 \le i \le n, 
               [a_k, a_j] = v_{jk}, 1 \le j \< k \le n \}.  
$$
where $v_{jk}$ is a word in the elements $a_{k+1},\dots,a_n$ for 
$1 \le j \< k \le n$.

\index{power-commutator presentation}\index{pc presentation}\index{pcp}
\index{pc generators}\index{collection}
This is called a *power-commutator* presentation (or *pc presentation*
or *pcp*) of $G$, generators from such a presentation will be referred
to as *pc generators*. In terms of such pc generators every element of
$G$ can be written in a  ``normal  form''  $a_1^{e_1},\dots,a_n^{e_n}$
with $0 \le e_i \< p$. Moreover any given product  of  the  generators
and their inverses can be brought into such a normal  form  using  the
defining relations in the above presentation  as  rewrite  rules.  Any
such process is called *collection*. For  the  discussion  of  various
collection methods see \cite{LGS90} and 
\cite{VL90a}. %this should be Collection

\index{consistent}\index{confluent rewriting system}\index{confluent}
Every $p$-group of order $p^n$ has such  a  pcp  presentation  on  $n$
generators and conversely every such presentation defines a $p$-group.
However a $p$-group defined by a pc-presentation on $n$ generators can
be of smaller order $p^m$ with $m\<n$. A pcp on  $n$  generators  that
does in fact define a $p$-group of order $p^n$ is called  *consistent*
in this manual, in line with most of the literature on the  algorithms
occurring here. A consistent pcp  determines  a  *confluent  rewriting
system* (see~"ref:IsConfluent" in the {\GAP} Reference Manual) for the
group it defines and for this  reason  often  (in  particular  in  the
{\GAP} Reference Manual)  such  a  pcp  presentation  is  also  called
*confluent*.

Consistency of a pcp is tantamount to the fact that for any given word
in the generators any two collections will yield the same normal form.

\index{consistency conditions}
Consistency of a  pcp can be checked by a  finite set of *consistency
conditions*, demanding  that collection of the left hand  side and of
the right  hand side of  certain equations, starting  with subproducts
indicated by bracketing, will result in the same normal form.  There
are 3 types of such equations (that will be referred to in the manual):
$$
\matrix{
(a^n)a &=& a(a^n)                                \hfill&{\rm (Type\ 1)}\cr
(b^n)a &=& b^{(n-1)}(ba), b(a^n) = (ba)a^{(n-1)} \hfill&{\rm (Type\ 2)}\cr
 c(ba) &=& (cb)a                                 \hfill&{\rm (Type\ 3)}\cr
}
$$

*Exponent-$p$ Central Series and Weighted pc Presentations*

For details, see \cite{NNN98}.

\atindex{exponent-p central series}{@exponent-$p$ central series}
The (*descending*  or  *lower*)  (*exponent-*)*$p$-central series* 
of an arbitrary  group $G$ is defined by  
$$
P_1(G)  := G,  P_i(G) := [G, P_{i-1}(G)] P_{i-1}(G)^p\.
$$
For a $p$-group $G$ this  series terminates with the trivial group. $G$
\index{class}\atindex{p-class}{@$p$-class}
has  *$p$-class*  $c$  if  $c$  is  the  smallest  integer  such  that
$P_{c+1}(G)$ is the trivial group. In this manual, as well as in  much
of the literature about the `pq'- and related algorithms the $p$-class
is often referred to simply as *class*.

Let  the  $p$-group $G$  have  a consistent  pcp  as  above. Then  the
subgroups
$$
\langle1\rangle \< {\langle}a_1\rangle \< {\langle}a_1, a_2\rangle %
    \< \dots \< {\langle}a_1,\dots,a_i\rangle \< \dots \< G
$$
form a central series  of $G$. If this refines  the $p$-central series,
\index{weight function}
we can define the *weight function*  $w$  for  the  pc  generators  by
$w(a_i) = k$  if  $a_i$  is  contained  in  $P_{k-1}(G)$  but  not  in
$P_k(G)$.

\index{weighted pcp}
The pair of  such a weight function and  a pcp allowing it  is called a
*weighted pcp*.

*$p$-Cover, $p$-Multiplicator*

For details, see \cite{NNN98}.

\atindex{p-covering group}{@$p$-covering group}\atindex{p-cover}{@$p$-cover}
\atindex{p-multiplicator}{@$p$-multiplicator}
\atindex{p-multiplicator rank}{@$p$-multiplicator rank}
\index{multiplicator rank}
Let $d$  be the minimal number  of generators of the  $p$-group $G$ of
$p$-class $c$.   Then $G$ is isomorphic  to a factor group  $F/R$ of a
free group $F$ of  rank $d$. We denote $[F, R] R^p$  by $R^\*$.  It can
be proved (see e.g.~\cite{OBr90}) that the isomorphism type of $G^\* :=
F/R^\*$ depends only on $G$.   $G^\*$ is called the *$p$-covering group*
or *$p$-cover* of $G$, and  $R/R^\*$ the $p$-multiplicator of $G$.  The
*$p$-multiplicator* is  of course  an elementary abelian  $p$-group, its
minimal number of generators is called the *($p$-)multiplicator rank*.

*Descendants, Capable, Terminal, Nucleus*

For details, see \cite{New77} and  \cite{OBr90}.

\index{descendant}\index{immediate descendant}\index{nucleus}
\index{capable}\index{terminal}\index{allowable}\index{permutation}
\index{extended automorphism}
Let again $G$  be a $p$-group of $p$-class $c$ and  $d$ the
minimal  number  of   generators  of  $G$.   A  $p$-group   $H$  is  a
*descendant* of $G$ if the minimal  number of generators of $H$ is $d$
and $H/P_c(H)$  is isomorphic to $G$.   A descendant $H$ of  $G$ is an
*immediate  descendant* if  it  has $p$-class  $c+1$.   $G$ is  called
*capable* if it has immediate descendants, otherwise *terminal*.

Let $G^\* = F/R^\*$ again be  the  $p$-cover  of  $G$,  Then  the  group
$P_c(G^\*)$ is called the *nucleus* of $G$.  Note  that  $P_c(G^\*)$  is
contained in the $p$-multiplicator $R/R^\*$.

It is  proved e.g. in  \cite{OBr90} that the immediate  descendants of
$G$  are  obtained  as  factorgroups  of  the  $p$-cover  by  (proper)
supplements   of    the   nucleus   in    the   (elementary   abelian)
$p$-multiplicator. These are also called *allowable*.

\index{extended automorphism}\index{permutations}
It is further proved there that every automorphism $\alpha$  of  $F/R$
extends to an autmorphism $\alpha^\*$ of the $p$-cover $F/R^\*$ and that
the restriction of $\alpha^\*$ to the multiplicator $R/R^\*$ is uniquely
determined  by  $\alpha$.  Each  *extended  automorphism*   $\alpha^\*$
induces a permutation of the allowable subgroups.  Thus  the  extended
automorphisms determine a group $P$ of *permutations* on  the  set  of
$A$ of allowable subgroups. Choosing a representative  $S$  from  each
orbit of $P$ on $A$, the set of  factor  groups  $F/S$  contains  each
(isomorphism type of) immediate descendant of $G$ exactly once.

(The group $P$ of permutations will appear in the description  of
some interactive functions.)


%%%%%%%%%%%%%%%%%%%%%%%%%%%%%%%%%%%%%%%%%%%%%%%%%%%%%%%%%%%%%%%%%%%%%
\Section{The p-quotient Algorithm}

For details see \cite{HN80} and \cite{NO96}. Other descriptions of the
algorithm are given in \cite{VL90b} and \cite{Sims94}. %not collection

\index{p-quotient algorithm}
The aim of the $p$-quotient (`pq'-) algorithm is  to  find  $p$-factor
groups (or $p$-quotient groups) of a  given  finitely  presented  (fp)
group $G$ for a given prime $p$.

The `pq'  algorithm successively determines  the factor groups  of the
groups of the $p$-central series of a fp group $G$. If a bound $b$ for
the  $p$-class is  given, the  algorithm will  determine  those factor
groups  up  to  at  most  $p$-class $b$,  If  the  $p$-central  series
terminates with a subgroup $P_k(G)$  with $k \< b$, the algorithm will
stop with that group.  If no such  bound is given, it will try to find
the biggest such factorgroup.

$G/P_2(G)$ is  the largest elementary abelian $p$-factor  group of $G$
and this  can be found  from the relation  matrix of $G$  using matrix
diagonalisation   modulo  $p$.   So   it  suffices   to  explain   how
$G/P_{i+1}(G)$ is found from $G$ and $G/P_i(G)$ for some $i \ge 1$.


*UP TO HERE*

If they are  not fulfilled they will provide a  set of equations among
the 

%%%%%%%%%%%%%%%%%%%%%%%%%%%%%%%%%%%%%%%%%%%%%%%%%%%%%%%%%%%%%%%%%%%%%
\Section{Standard Presentation and Isomorphism Testing}

For details see \cite{OBr94}.


%%%%%%%%%%%%%%%%%%%%%%%%%%%%%%%%%%%%%%%%%%%%%%%%%%%%%%%%%%%%%%%%%%%%%
\Section{The p-group generation Algorithm}

For details see \cite{New77} and \cite{OBr90}.


%%%%%%%%%%%%%%%%%%%%%%%%%%%%%%%%%%%%%%%%%%%%%%%%%%%%%%%%%%%%%%%%%%%%%
\Section{Old stuff}

\beginlist

\item{1.}
A $p$-quotient  algorithm  to  compute  pc-presentations  for  $p$-factor
groups of finitely presented groups. The algorithm  implemented  here  is
based on that described by Newman  and  O'Brien,  Havas  and
Newman, and papers referred to there. 
A FORTRAN implementation of this algorithm was programmed by  Alford  and
Havas. The basic data structures of that implementation are retained.

\item{2.} 
A $p$-group generation algorithm to generate pc-presentations  of  groups
of prime power order. The algorithm implemented  here  is  based  on  the
algorithms described by Newman.  A
FORTRAN implementation of this algorithm was developed earlier by  Newman
and O'Brien.

\item{3.}
A  standard  presentation  algorithm  used   to   compute   a   canonical
pc-presentation  of  a  $p$-group.  The  algorithm  implemented  here  is
described in.

\item{4.} 
An algorithm which can be used to compute the  automorphism  group  of  a
$p$-group. The algorithm implemented here is  described  in~\cite{OBr94}.
This part of  the  standalone  program  is  not  accessible  through  the
{\ANUPQ} package. Instead, users are advised  to  consider  the  {\GAP}~4
package {\AutPGrp}, which implements a better algorithm in {\GAP} for the
computation of automorphism groups of $p$-groups.

\endlist

Further   background   may   be   found   in~\cite{OBr95},   \cite{VL84},
\cite{NNN98}.

%%%%%%%%%%%%%%%%%%%%%%%%%%%%%%%%%%%%%%%%%%%%%%%%%%%%%%%%%%%%%%%%%%%%%%%%%
%%
%E
