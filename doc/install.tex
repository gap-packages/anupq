%%%%%%%%%%%%%%%%%%%%%%%%%%%%%%%%%%%%%%%%%%%%%%%%%%%%%%%%%%%%%%%%%%%%%%%%%
%%
%W  install.tex     ANUPQ documentation - installation      Werner Nickel
%%
%%
%H  $Id$
%%
%%

%%%%%%%%%%%%%%%%%%%%%%%%%%%%%%%%%%%%%%%%%%%%%%%%%%%%%%%%%%%%%%%%%%%%%%%%%%%%%
\Chapter{Installing the ANUPQ Share Package}

The ANU `pq' binary is written in C and the package can only be installed
under UNIX. It has  been  tested  on  DECstation  running  Ultrix,  a  HP
9000/700 and HP 9000/800 running HP-UX, a MIPS running RISC/os  Berkeley,
a NeXTstation running NeXTSTEP 3.0, a SUN  running  SunOS  and  an  Intel
Pentium based PC running Linux.

To install the {\ANUPQ} share package, move the file `anupq.zoo' into the
`pkg' directory in which you plan to install {\ANUPQ}. Usually, this will
be the directory `pkg' in  the  hierarchy  of  your  version  of  GAP  4.
(However, it is also possible to keep an additional  `pkg'  directory  in
your private directories, see section "ref:Installing Share Packages"  of
the {\GAP}~4 reference manual for details on how to do this.)

Then unzoo `anupq.zoo' by

\begintt
unzoo -x anupq
\endtt

Change directory to the newly created `anupq' directory. Now you need  to
call `configure <path>' where <path> is  the  path  to  the  {\GAP}  home
directory. So for example if you install the package in  the  main  `pkg'
directory call

\begintt
./configure ../..
\endtt

This will fetch the architecture type for which {\GAP} has been  compiled
last, create a `Makefile' and list a number of ``targets'' to call `make'
with. If you have one of  the  standard  Linux  (or  NetBSD  or  FreeBSD)
systems with `gcc', wish to compile with  `-O2'  optimisation,  and  have
`gmp' with its include and  library  files  in  `/usr/local/include'  and
`/usr/local/lib', respectively, you can now simply call

\begintt
make
\endtt

to compile the binary and to install it in the appropriate place.

If you need a special target (perhaps you don't have `gmp' or you are not
on a Linux, NetBSD or FreeBSD system) then you need to call `make' with a
target argument. If  the  targets  displayed  on  the  screen  after  the
`configure' step rushed past your eyes and you can't scroll back  to  see
them, you can ``pipe'' those same targets through `less' or `more',  e.g.
with `more':

\begintt
make unknown || more
\endtt

An abbreviation of the target list is as follows:

\begintt
'linux-iX86-gcc2-gmp'      for IBM x86 PCs under linux/BSD with GNU cc 2 and mp
'linux-iX86-cc-gmp'        for IBM x86 PCs under linux/BSD with cc and GNU mp
'linux-iX86-gcc2'          for IBM x86 PCs under linux/BSD with GNU cc 2
'linux-iX86-cc'            for IBM x86 PCs under linux/BSD with cc (GNU)
[... 16 lines deleted ...]
'sunos-gcc2-gmp'           for SunOS with GNU cc 2 and gmp
'sunos-cc-gmp'             for SunOS with cc and GNU mp
'sunos-gcc2'               for SunOS with GNU cc 2
'sunos-cc'                 for SunOS with cc
'unix-gmp'                 for a generic unix system with cc and GNU mp
'unix'                     for a generic unix system with cc
'clean'                    remove all created files

   targets are listed according to preference,
   i.e., 'sunos-gcc2' is better than 'sunos-cc'
   no target is the same as choosing 'linux-iX86-gcc2-gmp'
   additional C compiler and linker flags can be passed with
   'make <target> COPTS=<compiler-opts> LOPTS=<linker-opts>',
   e.g., 'make sunos-cc COPTS=-g LOPTS=-g'.

   set GAP if GAP4 is not started with the command 'gap',
   e.g., 'make sunos-cc GAP=/usr/local/bin/gap4'.

   in order to use the GNU multiple precision (gmp) set
   'GNUINC' (default '/usr/local/include') and 
   'GNULIB' (default '/usr/local/lib')

   do 'make unknown | more' to see these targets again via more
\endtt

Let's suppose that the  `linux-iX86-gcc2-gmp'  target  does  not  satisfy  your
requirements; let's suppose your system is Solaris 2.8 (i.e.~SunOS  5.8),
you have `gmp' but its include  and  library  directories  are  somewhere
else, and that `gap4' is the command used to initiate  {\GAP}~4.  Then  the
following `make' call might be correct in this case:

\begintt
make sunos-gcc2-gmp GAP=gap4 GNUINC=/opt/local/include GNULIB=/opt/local/lib
\endtt

If you  don't  have  the  *GNU*  multiple  precision  arithmetic  (`gmp')
installed on your system, not to worry, `gmp'  is  *not  required*;  just
select an appropriate target without `-gmp'.

Now it is time to test the installation. The  prompt  ```me@mypc >'''  is
supposed to represent your UNIX prompt.

*these tests will be replaced with a pqTest script*

The first test will only test the ANU `pq' binary.

\begintt
me@mypc > cd ../..
me@mypc > bin/i686-pc-linux-gnu-gcc/pq < gap/tst/test1.pga
# a lot of messages ending in
**************************************************
Starting group: c3c3 # 2;2 # 4;3
Order: 3^7
Nuclear rank: 3
3-multiplicator rank: 4
# of immediate descendants of order 3^8 is 7
# of capable immediate descendants is 5

**************************************************
34 capable groups saved on file c3c3_class4
Construction of descendants took 1.92 seconds

Select option: 0 
Exiting from p-group generation

Select option: 0 
Exiting from ANU p-Quotient Program
Total user time in seconds is 1.97
me@mypc > ls -l c3c3*
total 89
-rw-r--r--    1 gap    3320 Jun 24 11:24 c3c3_class2
-rw-r--r--    1 gap    5912 Jun 24 11:24 c3c3_class3
-rw-r--r--    1 gap   56184 Jun 24 11:24 c3c3_class4
me@mypc > rm c3c3_class*
\endtt

The second test will test the stacksize. If it is too small you will  get
a memory fault.

\begintt
me@mypc > bin/i686-pc-linux-gnu-gcc/pq < gap/tst/test2.pga
# a lot of messages ending in
**************************************************
Starting group: c2c2 # 1;1 # 1;1 # 1;1
Order: 2^5
Nuclear rank: 1
2-multiplicator rank: 3
Group c2c2 # 1;1 # 1;1 # 1;1 is an invalid starting group

**************************************************
Starting group: c2c2 # 2;1 # 1;1 # 1;1
Order: 2^5
Nuclear rank: 1
2-multiplicator rank: 3
Group c2c2 # 2;1 # 1;1 # 1;1 is an invalid starting group
Construction of descendants took 0.47 seconds

Select option: 0 
Exiting from p-group generation

Select option: 0 
Exiting from ANU p-Quotient Program
Total user time in seconds is 0.50
me@mypc > ls -l c2c2*
total 45
-rw-r--r--    1 gap   6228 Jun 24 11:25 c2c2_class2
-rw-r--r--    1 gap  11156 Jun 24 11:25 c2c2_class3
-rw-r--r--    1 gap   2248 Jun 24 11:25 c2c2_class4
-rw-r--r--    1 gap      0 Jun 24 11:25 c2c2_class5
me@mypc > rm c2c2_class*
\endtt

The third example tests the link between the ANU `pq' binary and  {\GAP}.
If there is a problem you will get  a  error  message  saying  `Error  in
system call to GAP'; if this  happens,  check  the  environment  variable
`ANUPQ_GAP_EXEC' (and fix it to the  correct  path  for  {\GAP}  if  it's
wrong).

\begintt
me@mypc > bin/i686-pc-linux-gnu-gcc/pq < gap/tst/test3.pga
[..messages from the ANU pq..]
**************************************************
Starting group: c5c5 # 1;1 # 1;1
Order: 5^4
Nuclear rank: 1
5-multiplicator rank: 2
# of immediate descendants of order 5^5 is 2

**************************************************
Starting group: c5c5 # 1;1 # 2;2
Order: 5^5
Nuclear rank: 3
5-multiplicator rank: 3
# of immediate descendants of order 5^6 is 3
# of immediate descendants of order 5^7 is 3
# of capable immediate descendants is 1
# of immediate descendants of order 5^8 is 1
# of capable immediate descendants is 1

**************************************************
2 capable groups saved on file c5c5_class4

**************************************************
Starting group: c5c5 # 1;1 # 2;2 # 4;2
Order: 5^7
Nuclear rank: 1
5-multiplicator rank: 2
# of immediate descendants of order 5^8 is 2
# of capable immediate descendants is 2

**************************************************
Starting group: c5c5 # 1;1 # 2;2 # 7;3
Order: 5^8
Nuclear rank: 2
# of immediate descendants of order 5^9 is 1
# of capable immediate descendants is 1
# of immediate descendants of order 5^10 is 1
# of capable immediate descendants is 1

**************************************************
4 capable groups saved on file c5c5_class5
Construction of descendants took 0.62 seconds

Select option: 0 
Exiting from p-group generation

Select option: 0 
Exiting from ANU p-Quotient Program
Total user time in seconds is 0.68
me@mypc > ls -l c5c5*
total 41
-rw-r--r--    1 gap     924 Jun 24 11:27 c5c5_class2
-rw-r--r--    1 gap    2220 Jun 24 11:28 c5c5_class3
-rw-r--r--    1 gap    3192 Jun 24 11:30 c5c5_class4
-rw-r--r--    1 gap    7476 Jun 24 11:32 c5c5_class5
me@mypc > rm c5c5_class*
\endtt

The fourth test will test the standard  presentation  part  of  the  `pq'
binary.

\begintt
me@mypc > bin/pq -i -k < gap/tst/test4.sp
[..messages from the ANU pq..]
Computing standard presentation for class 5 took 0.03 seconds

Select option: 0 
Exiting from ANU p-Quotient Program
Total user time in seconds is 6.97
me@mypc > ls -l SPRES
-rw-r--r--    1 me   mygroup        488 Mar 27 11:40 SPRES
me@mypc > diff SPRES gap/out4.sp
# there should be no difference if compiled with '-gmp'
156250000
me@mypc > rm SPRES
\endtt

The last test will test the link between {\GAP} and the ANU `pq'  binary.
If everything goes well you should not see any message.

\begintt
me@mypc > gap -b
gap> SetInfoLevel(InfoWarning, 0); RequirePackage( "anupq" );;
gap> SetInfoLevel(InfoWarning, 1);
gap> ReadTest( "gap/tst/anupga.tst" );
gap>
\endtt

%%%%%%%%%%%%%%%%%%%%%%%%%%%%%%%%%%%%%%%%%%%%%%%%%%%%%%%%%%%%%%%%%%%%%%%%%
%%
%E
