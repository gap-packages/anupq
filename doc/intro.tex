%%%%%%%%%%%%%%%%%%%%%%%%%%%%%%%%%%%%%%%%%%%%%%%%%%%%%%%%%%%%%%%%%%%%%%%%%
%%
%W  intro.tex       ANUPQ documentation - introduction      Werner Nickel
%%
%%
%H  $Id$
%%
%%

%%%%%%%%%%%%%%%%%%%%%%%%%%%%%%%%%%%%%%%%%%%%%%%%%%%%%%%%%%%%%%%%%%%%%%%%%
\Chapter{Introduction}

The {\GAP} 4 share package {\ANUPQ} provides an interface to the ANU PQ C
progam `pq' written by Eamonn O'Brien, making the functionality of the  C
program available to {\GAP}. Henceforth, we shall refer to  the  {\ANUPQ}
share package  (resp.  the  `pq'  binary),  when  discussing  the  {\GAP}
component (resp. the C code component) of ANU PQ. The ANU `pq' standalone
provides access to implementations of the following algorithms:

\beginlist

\item{1.}
A $p$-quotient algorithm to compute  power-commutator  presentations  for
$p$-factor groups of finitely presented groups. The algorithm implemented
here is based on that described by Newman and O'Brien \cite{NO96},  Havas
and Newman~\cite{HN80}, and papers referred to there. Another description
of  the   algorithm   is   given   by   Vaughan-Lee   (see~\cite{Vau90a},
\cite{Vau90b}). A FORTRAN implementation of this algorithm was programmed
by Alford and Havas. The basic data structures of that implementation are
retained.

\item{2.} 
A   $p$-group   generation   algorithm   to   generate   power-commutator
presentations of groups of prime power order. The  algorithm  implemented
here is based on the  algorithms  described  by  Newman~\cite{New77}  and
O'Brien~\cite{OBr90}. A FORTRAN  implementation  of  this  algorithm  was
developed earlier by Newman and O'Brien.

\item{3.}
A  standard  presentation  algorithm  used   to   compute   a   canonical
power-commutator presentation of a $p$-group. The  algorithm  implemented
here is described in~\cite{OBr94}.

\item{4.} 
An algorithm which can be used to compute the  automorphism  group  of  a
$p$-group. The algorithm implemented here is  described  in~\cite{OBr94}.
This part of  the  standalone  program  is  not  accessible  through  the
{\ANUPQ} share package.  Instead,  users  are  advised  to  consider  the
{\GAP}~4 share package {\AutPGrp}, which implements a better algorithm in
{\GAP} for the computation of automorphism groups of $p$-groups.

\endlist

Further   background   may   be   found   in~\cite{OBr95},   \cite{Vau84}
and~\cite{NNN98}.

For details regarding the standalone version see the file `guide.dvi'.

%%%%%%%%%%%%%%%%%%%%%%%%%%%%%%%%%%%%%%%%%%%%%%%%%%%%%%%%%%%%%%%%%%%%%%%%%
\Section{How to read this manual}

\index{ANUPQ}
Instructions on how to install and load the {\ANUPQ}  share  package  are
located are located in  Sections~"Installing  the  ANUPQ  Share  Package"
and~"Loading the ANUPQ Share Package", respectively.

*need to fill in some details here about the structure of the manual
 or perhaps this section should come first*

%%%%%%%%%%%%%%%%%%%%%%%%%%%%%%%%%%%%%%%%%%%%%%%%%%%%%%%%%%%%%%%%%%%%%%%%%
\Section{The ANUPQData Record}

\>`ANUPQData' V

is a {\GAP} record in which the essential data for  an  {\ANUPQ}  session
within {\GAP} is stored; its fields are:

\beginitems

\quad`binary' & the path of the {\ANUPQ} binary;

\quad`tmpdir' & the  path  of  the  temporary  directory  containing  the
non-interactive   {\ANUPQ}   input   and   output   files    (also    see
"ANUPQDirectoryTemporary" below);

\quad`io' & list of data records for `PqStart' (see below  and~"PqStart")
processes;

\quad`infile' & the full path of the (non-interactive) `pq' input file;

\quad`outfile'& the full path of the default `pq' output  file;

\quad`outfname' & is the file to which `pq' output is directed, which  is
`ANUPQData.outfile' except when option `SetupFile' is used, in which case
`outfname' is set to `"PQ_OUTPUT"';

and

\quad`version'& the version of the current `pq' binary.

\enditems

Some or all of the fields: `group', `workspace', `menu', `pQuotient'  and
`pQepi' may also be set after  a  non-interactive  {\ANUPQ}  function  is
called.  They  have  a  similar  role  as  the  corresponding  fields  of
`ANUPQData.io[<ioIndex>]' we describe below.

Each time an interactive {\ANUPQ}  process  is  initiated  via  `PqStart'
(see~"PqStart"), an identifying number <ioIndex>  is  generated  for  the
interactive process  and  a  record  `ANUPQData.io[<ioIndex>]'  with  the
following fields is created:

\beginitems

\quad`stream'& the  IOStream  opened  for  interactive  {\ANUPQ}  process
<ioIndex>;

\quad`group'& the group given as first argument to `PqStart';

\quad`workspace'& the workspace set for the `pq' process (either given as
a second argument to `PqStart', or set by default to 10000000);

\quad`menu'& the current menu of the `pq' process  (the  `pq'  binary  is
managed by various  menus,  the  details  of  which  the  user  shouldn't
normally need to know about -- the `menu' field remembers which menu  the
`pq' process is currently ``in'');

\quad`outfname' & is the file to which `pq' output is directed, which  is
always `ANUPQData.outfile' (see above) for an interactive process;

\quad`pQpcp' & If this field is bound  it  is  a  record  containing  the
options used the last time that  `pq'  computed  a  pc  presentation  for
`group', while computing a $p$-quotient i.e.~if it is  bound  then  there
has been such a call. (It is used  mainly  to  avoid  re-computing  a  pc
presentation that's been done already, by functions that require  that  a
pc presentation is known to the `pq' binary.)

\quad`pQuotient'  &  This  is  set  to  the  value   returned   by   `Pq'
(see~"Pq!interactive") when called interactively, for process  <i>.  (The
field `pQepi' is also set at the same time.)

\quad`pQepi' & This is set  to  the  value  returned  by  `PqEpimorphism'
(see~"PqEpimorphism!interactive") when called interactively, for  process
<i>. (The field `pQuotient' is also set at the same time.)

\quad`SPpcp' & If this field is bound  it  is  a  record  containing  the
options used the last time that  `pq'  computed  a  pc  presentation  for
`group', while computing a standard presentation i.e.~if it is bound then
there has been such a call. (It is used mainly to avoid re-computing a pc
presentation that's been done already, by functions that require  that  a
pc presentation is known to the `pq' binary.)

\quad`SP' & This is set to the value returned by `PqStandardPresentation'
or `StandardPresentation' (see~"PqStandardPresentation!interactive") when
called interactively, for process <i>. (The field `SPepi' is also set  at
the same time.)

\quad`SPepi'   &   This   is   set   to    the    value    returned    by
`EpimorphismPqStandardPresentation' or  `EpimorphismStandardPresentation'
(see~"EpimorphismPqStandardPresentation!interactive")     when     called
interactively, for process <i>. (The field `SP' is also set at  the  same
time.)

\enditems

\>ANUPQDirectoryTemporary( <dir> ) F

calls the UNIX command `mkdir' to create <dir>, which must be  a  string,
and if successful a directory  object  for  <dir>  is  both  assigned  to
`ANUPQData.tmpdir'  and  returned.  The  fields  `ANUPQData.infile'   and
`ANUPQData.outfile' are also set to be files in  `ANUPQData.tmpdir',  and
on exit from {\GAP} <dir> is removed. Most users  will  never  need  this
command; by default, {\GAP} typically chooses a  ``random''  subdirectory
of `/tmp' for `ANUPQData.tmpdir' which may occasionally  have  limits  on
what may be written there. `ANUPQDirectoryTemporary' permits the user  to
choose a directory (object) where one is not so limited.

%%%%%%%%%%%%%%%%%%%%%%%%%%%%%%%%%%%%%%%%%%%%%%%%%%%%%%%%%%%%%%%%%%%%%%%%%
\Section{Setting the Verbosity of ANUPQ via Info and InfoANUPQ}

\>`InfoANUPQ' V

The input to and the output from the `pq'  binary  is,  by  default,  not
displayed. However the user may choose to  see  some,  or  all,  of  this
input/output.   This   is   done   via   the   `Info'   mechanism    (see
Chapter~"ref:Info Functions" in the {\GAP} Reference  Manual).  For  this
purpose,  there  is  the  <InfoClass>  `InfoANUPQ'.  Each  line  of  `pq'
input/output is directed to a call to `Info' and will  be  displayed  for
the user to see if the `InfoLevel' of  `InfoANUPQ'  is  high  enough.  By
default, the `InfoLevel' of `InfoANUPQ' is 1, and it is recommended  that
you leave it at this level, or higher. Only messages which we think  that
the user will really want to see are directed to  `Info'  at  `InfoANUPQ'
level 1. To turn off *all* `InfoANUPQ'  messaging,  set  the  `InfoANUPQ'
level to 0.

Currently, information from the `pq' binary is directed to `Info' at four
`InfoANUPQ' levels: 1, 2, 3 and 4. The command

\beginexample
gap> SetInfoLevel(InfoANUPQ, 2);
\endexample

enables the display of results from the `pq'  binary.

\beginexample
gap> SetInfoLevel(InfoANUPQ, 3);
\endexample

enables the display of all the input sent to the `pq'  binary,  behind  a
```ToPQ> ''' prompt (so that you can distinguish it from the output  from
the `pq' binary). Finally,

\beginexample
gap> SetInfoLevel(InfoANUPQ, 4);
\endexample

enables the display of all other output from the `pq' binary, namely  the
banner, menus, and the timing data printed when the `pq' binary exits.

%%%%%%%%%%%%%%%%%%%%%%%%%%%%%%%%%%%%%%%%%%%%%%%%%%%%%%%%%%%%%%%%%%%%%%%%%
\Section{Authors}

The C implementation of the ANU `pq' standalone was developed by

\begintt
Eamonn O'Brien
Department of Mathematics
University of Auckland
Private Bag 92019
Auckland
New Zealand
\endtt

{\kernttindent}`email:' \Mailto{obrien@math.auckland.ac.nz}

The {\GAP} 4 version of this package was adapted from the {\GAP} 3
version by  

\begintt
Werner Nickel
AG 2, Fachbereich Mathematik, TU Darmstadt
Schlossgartenstr. 7, 64289 Darmstadt, Germany
\endtt

{\kernttindent}`email:' \Mailto{nickel@mathematik.tu-darmstadt.de}

%%%%%%%%%%%%%%%%%%%%%%%%%%%%%%%%%%%%%%%%%%%%%%%%%%%%%%%%%%%%%%%%%%%%%%%%%
%%
%E
